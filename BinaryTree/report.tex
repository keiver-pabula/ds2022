\documentclass{ctexart}
\usepackage[utf8]{inputenc}
\usepackage{graphicx}
\usepackage{tikz}
\usetikzlibrary{shapes,arrows}


\title{作业4: 二叉树和节点的逻辑设计}
\author{陈科辉 Keiver Pabula}
\date{04 November,2022}
\begin{document}
\maketitle

\section{设计思路}
  首先对于各类的二叉树,主要所需要的函数就是:构造函数,插入,删除,判断值是否在书内,最大和最小值,判断树是否为空,清空树,遍历打印树内的值等功能,所以我将这些功能都放入头文件内。
  二叉树的节点(Node)在树中就是包含了一个数据元素,并且只想子树的分支。并且我将结点放置在树类的外面,独自作为一个类。

\section{二叉搜索树}
  二叉搜索树简单来说就是他有一个根,然后每个父节点最多有两个子节点,所以子节点可能有0,1,2个;对于每个节点,它的下一个子节点中,节点必须小于它的父节点,并且左节点要小于右节点的值,这样做的目的就是为了更容易做排序。 实现方法就是使用递归,逐步添加一个节点,然后寻找对应的位置,小于则向左走,大于则向右走。
\section{平衡树}
  平衡数大致与二叉搜索树一致但是它解决了二叉搜索树深度不统一的问题,所以它的复杂度会比二叉搜索数更好。
\section{关系和区别}
首先BinarySearchTree和AvlTree都是BinaryTree的一种,它们两者都继承了BinaryTree的所有性质,并且他们有增加的性质。
而平衡树和二叉搜索数其实也类似,他们主要的区别就在于他们的深度以及所带来的影响“复杂度”。相比于二叉搜索树,平衡树更严格的点就在于它的插入和删除函数,它们需要保证它深度的问题,左子数和右子树不接受>1的差距。


\end{document}
